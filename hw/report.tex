\documentclass{ctexart}
\usepackage{amsmath,bm}
\usepackage{setspace}
\usepackage{xeCJK}
\usepackage{xcolor}
\usepackage{indentfirst}
\usepackage{listings}
\usepackage{graphicx}
\usepackage{subfigure}
\usepackage{amsfonts,amssymb}
\usepackage[a4paper,scale=0.8]{geometry}
\usepackage{hyperref}
\usepackage{float}
\usepackage{changepage}
\usepackage{xcolor}
\usepackage{listings}
\lstset{
basicstyle=\small,%
escapeinside=``,%
keywordstyle=\color{blue} \bfseries,% \underbar,%
identifierstyle={},%
commentstyle=\color{green},%
stringstyle=\ttfamily,%
%labelstyle=\tiny,%
extendedchars=false,%
linewidth=\textwidth,%
numbers=left,%
numberstyle=\tiny \color{blue},%
frame=trbl%
}


\lstset{breaklines}%这条命令可以让LaTeX自动将长的代码行换行排版

\lstset{extendedchars=false}%这一条命令可以解决代码跨页时,章节标题,页眉等汉字不显示的问题
\setCJKmainfont{华光书宋_CNKI}
\newCJKfontfamily\kaiti{华光楷体_CNKI}
\newCJKfontfamily\hei{华光黑体_CNKI}
\newCJKfontfamily\fsong{华光仿宋_CNKI}
\newfontfamily\code{Courier New}
\linespread{1.5} \setlength\parindent{2 em}
\title{\Huge 中国科学技术大学计算机学院\\《计算机体系结构》作业}
\date{\LARGE 2021.06.07}
\begin{document}
\begin{hei}  \maketitle\end{hei}
\begin{figure}[htbp]
    \centering
    \includegraphics[scale=0.4]{USTC.png}

\end{figure}
\begin{LARGE}\begin{align*}            & \text{作业题目:\underline{第六章作业}} \\
         & \text{学生姓名:\underline{胡毅翔}}     \\
         & \text{学生学号:\underline{PB18000290}}\end{align*}\end{LARGE}
\par
\par\par
\centerline{\large 计算机实验教学中心制}
\par \centerline {\large 2019年9月}
\newpage
\begin{enumerate}
    \item[第一题]
          在以下的循环中,找出所有真相关、输出相关和反相关。通过重命名来消除输出相关和反相关。
          \begin{lstlisting}[language=C]
for(i = 0; i < 100; i++){
    A[i] = A[i] * B[i];  /* S1 */
    B[i] = A[i] + c;     /* S2 */
    A[i] = C[i] * c;     /* S3 */
    C[i] = D[i] * A[i];  /* S4 */
}
\end{lstlisting}
          \par 解:
          \par 真相关:S1到S2基于A[i],S3到S4基于A[i]。
          \par 输出相关:S1到S3基于A[i]。
          \par 反相关:S3到S4基于C[i]。
          \par 修改代码如下:
          \begin{lstlisting}[language=C]
for(i = 0; i < 100; i++){
    T[i] = A[i] * B[i];  
    B1[i] = T[i] + c;     
    A1[i] = C[i] * c;     
    C1[i] = D[i] * A1[i];  
}
\end{lstlisting}
    \item[第二题] 考 虑 以 下 代 码 , 它 将 两 个 包 含 单 精 度 复 数 值 的 向 量 相 乘:
          \begin{lstlisting}[language=C]
for ( i = 0; i < 300; i++){
    c_re[i] = a_re[i] * b_re[i] - a_im[i] * b_im[i];
    c_im[i] = a_re[i] * b_im[i] + a_im[i] * b_re[i];
}
\end{lstlisting}
          假定处理器的运行频率为 $700 \mathrm{MHz}$ ,最大向量长度为 64 。
          载入/存储单元的启动开销为 $15 $个时 钟周期,
          乘法单元为 $8$ 个时钟周期,加法/减法单元为 $5$ 个时钟周期。
          \begin{enumerate}
              \item 这个内核的运算密度为多少? 给出理由。
                    \par 解:
                    \par 这一代码每$6$个FLOP,读$4$个浮点数,写$2$个浮点数。所以,运算密度$=6/6=1$。
              \item 将此循环转换为使用条带挖掘的 $\mathrm{VMIPS}$ 汇编代码。
                    \par 解:
                    \par 假设$MVL=64$:
                    \begin{lstlisting}[language={[x86masm]Assembler}]
    li      $VL,44          # perform the first 44 ops
    li      $r1,0           # initialize index
loop:
    lv      $v1,a_re+$r1    # load a_re
    lv      $v3,b_re+$r1    # load b_re
    mulvv.s $v5,$v1,$v3     # a_re * b_re 
    lv      $v2,a_im+$r1    # load a_im
    lv      $v4,b_im+$r1    # load b_im
    mulvv.s $v6,$v2,$v4     # a_im * b_im   
    subvv.s $v5,$v5,$v6     # a_re * b_re - a_im * b_im
    sv      $v5,c_re+$r1    # store c_re
    mulvv.s $v5,$v1,$v4     # a_re * b_im
    mulvv.s $v6,$v2,$v3     # a_im * b_re
    addvv.s $v5,$v5,$v6     # a_re * b_im + a_im * b_re
    sv      $v5,c_im+$r1    # store c_im
    bne     $r1,0,else      # check if first iteration
    addi    $r1,$r1,#44     # first iteration, increment by 44
    j loop                  # guarabteed bext iteration
else:
    addi    $r1,$r1,#256    # not first iteration, increment by 256
skip:
    blt     $r1,1200,loop   # next iteration?
\end{lstlisting}
              \item  假定采用链接和单一存储器流水线,需要多少次钟鸣?
                    每个复数结果值需要多少 个时钟周期(包括启动开销在内)?
                    \par 解:
                    $$
                        \begin{array}{lll}
                            mulvv.s & lv      & \#\ a\_re\ *\ b\_re,\ load\ a\_im               \\
                            lv      & mulvv.s & \#\ load\ b\_im,\ a\_im\ *\ b\_im               \\
                            subvv.s & sv      & \#\ substract\ and\ store\ c\_re                \\
                            mulvv.s & lv      & \#\ a\_re\ *\ b\_im,\ load\ next\ a\_re\ vector \\
                            mulvv.s & lv      & \#\ a\_im\ *\ b\_re,\ load\ next\ b\_re\ vector \\
                            addvv.s & sv      & \#\ add\ and\ store\ c\_im
                        \end{array}
                    $$
                    \par 需要6次钟鸣。
                    \par 每次循环的周期数$=6\mathrm{chimes}\times 64\mathrm{elements}+15\mathrm{cycles(load/store)}
                        \times6+8\mathrm{cycles(multiply)}\times 4+5\mathrm{cycles(add/substract)}
                        \times 2=516$
                    \par 每个复数结果值需要的周期数$=516/128=4$
              \item 如果向量序列被链接在一起,每个复数结果值需要多少
                    个时钟周期 (包含开销 ) ?
                    \par 解:
                    \par 每次循环的周期数$=6\mathrm{chimes}\times 64\mathrm{elements}+15\mathrm{cycles(load/store)}
                        \times6+8\mathrm{cycles(multiply)}\times 4+5\mathrm{cycles(add/substract)}
                        \times 2=516$
                    \par 每个复数结果值需要的周期数$=516/128=4$
          \end{enumerate}

    \item[第三题] 假定有一种包含 $10$ 个 $\mathrm{SIMD}$ 处理器的
          $\mathrm{GPU}$ 体系结构。每条 $\mathrm{SIMD}$ 指令的宽度 为 $32$,
          每个 $\mathrm{SIMD}$处理器包含 $8$ 个车道,用于执行单精度运算和
          载入/存储指令,也就是说, 每个非分岔$\mathrm{SIMD}$指令每 $4$ 个时钟周期
          可以生成 $32$ 个结果。假定内核的分岔分支将导致平均 $80 \%$ 的线程为活
          动的。假定在所执行的全部 $\mathrm{SIMD}$ 指令中, $70 \%$ 为单精度运
          算、20\%为载人/ 存储。由于并不包含所有存储器延迟, 所以假定
          $\mathrm{SIMD}$ 指令平均发射率为 $0.85$ 。假定 $\mathrm{GPU}$ 的时
          钟速度为 $1.5 \mathrm{GHz}$。
          \begin{enumerate}
              \item 计算这个内核在这个 $\mathrm{GPU}$上的吞吐量,单位为 $\mathrm{GFLOP/s}$。
                    \par 解:
                    \par 吞吐量为$1.5\mathrm{GHz}\times 0.80\times 0.85\times 0.70\times 10cores\times32/4=57.12\mathrm{GFLOP/s}$。

              \item 假定我们有以下选项:\begin{enumerate}
                        \item 将单精度车道数增大至 $16$。
                        \item 将 $\mathrm{SIMD}$ 处理器数增大至 $15 $( 假定这一改变
                              不会影响所有其他性能度量,代码会扩展到增 加的处理器上)。
                        \item 添加缓存可以有效地将存储器延迟缩减 $40 \%$, 这样会将指令发射率增加至 $0.95$, 对于这些改
                              进中的每一项。


                    \end{enumerate}


                    \par 吞吐裏的加速比为多少?
                    \par 解:
                    \begin{enumerate}
                        \item $1.5\mathrm{GHz}\times 0.80\times 0.85\times 0.70\times 10\mathrm{cores}\times32/2=114.24\mathrm{GFLOP/s}$
                              \par $\mathrm{Speedup}=114.24/57.12=2$
                        \item $1.5\mathrm{GHz}\times 0.80\times 0.85\times 0.70\times 15\mathrm{cores}\times32/4=85.68\mathrm{GFLOP/s}$
                              \par $\mathrm{Speedup}=85.68/57.12=1.5$
                        \item $1.5\mathrm{GHz}\times 0.80\times 0.95\times 0.70\times 10\mathrm{cores}\times32/4= 63.84\mathrm{GFLOP/s}$
                              \par $\mathrm{Speedup}=63.84/57.12=1.11$
                    \end{enumerate}
          \end{enumerate}

    \item[第四题]  假定一个虚设 $\mathrm{GPU}$ 具有以下特性:
          \begin{itemize}
              \item 时钟频率为 $1.5 \mathrm{GHz}$;
              \item 包含 16 个 SIMD处理器,每个处理器包含 16 个单精度浮点单元;
              \item 片外存储器带宽为 $100 \mathrm{GB/s}$。
          \end{itemize}
          不考虑存储器带宽,假定所有存储器延迟可以隐藏,则这一 $\mathrm{GPU}$
          的峰值单精度浮点吞吐量为 多少 $\mathrm{GFLOP/s}$?
          在给定存储器带宽限制下,这一吞吐量是否可持续?
          \par 解:
          \par 该$\mathrm{GPU}$的峰值吞吐量为 $1.5 \times 16 \times 16=384\mathrm{GFLOP/s}$
          的单精度吞吐量。
          但是假设每个单精度操作都需要两个$4$字节操作数并输出一个$4$字节结果。维持
          这一吞吐量(结社没有时间局限性)将需要$12 \mathrm{bytes/FLOP} \times 384 \mathrm{GFLOP/s}=4.6\mathrm{TB/s}$的片外存储器带宽。
          因此,这种吞吐量是不可持续的,但当使用片上存储器时,仍然可以在短时间内实现。
\end{enumerate}
\end{document}
